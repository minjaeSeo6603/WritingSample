\documentclass{article}

% Language setting
\usepackage[english]{babel}
\usepackage{longtable}

% Set page size and margins
\usepackage[letterpaper, margin=1in]{geometry}

% Useful packages
\usepackage{amsmath}
\usepackage{graphicx}
\usepackage[colorlinks=true, allcolors=blue]{hyperref}
\usepackage{setspace}
\setstretch{1.5}
\usepackage{adjustbox}
\usepackage{booktabs}
\usepackage{array}
\usepackage[table]{xcolor}
\usepackage{colortbl}
\usepackage{siunitx}
\usepackage{float}
\usepackage{threeparttable}
\usepackage{multirow}

\title{\textbf{Which Groups Benefited Most from the 2006 Massachusetts Health Care Reform?}}
\author{\textbf{Minjae Seo}}
\date{}

\begin{document}
\maketitle

\begin{abstract}
\noindent This paper examines the heterogeneous effects of the 2006 Massachusetts Health Care Reform on health insurance coverage across demographic groups. Employing a difference-in-differences design with New England states as controls, I find that the reform reduced the uninsurance rate by approximately 5 percentage points, representing a 60 percent decline relative to the pre-reform mean. The effects are substantially larger for vulnerable populations: low-income individuals experienced an 11 percentage point reduction in uninsurance, while young adults (ages 18--30) saw a 12 percentage point decline. These findings are consistent with the reform's explicit targeting of these groups through subsidized coverage and the individual mandate. The parallel trends assumption is supported by pre-treatment coefficient estimates that are statistically indistinguishable from zero across all specifications.
\end{abstract}

\newpage
\section{Introduction}

The 2006 Massachusetts Health Care Reform represents one of the most significant state-level health policy experiments in U.S. history.\footnote{The reform was signed into law by Governor Mitt Romney on April 12, 2006, and took effect on July 1, 2007.} Enacted in response to the federal Deficit Reduction Act of 2005, the reform introduced a comprehensive framework to achieve near-universal health coverage, combining an individual mandate with subsidized insurance options for low- and moderate-income residents. The Massachusetts model subsequently served as the blueprint for the 2010 Affordable Care Act (ACA), making the evaluation of its effects crucial for understanding the potential impacts of similar policies nationwide.

The reform's key provisions included: (1) an individual mandate requiring residents to obtain health insurance or face tax penalties of up to 50\% of the lowest-cost premium;\footnote{The penalty was phased in gradually, reaching full implementation by 2008.} (2) the creation of the Commonwealth Health Insurance Connector, a health insurance exchange facilitating comparison shopping; (3) expanded Medicaid eligibility (MassHealth) for individuals up to 300\% of the Federal Poverty Level (FPL); and (4) employer requirements to offer health insurance or pay annual assessments of up to \$295 per employee for firms with 11 or more workers.

Subsidies were structured progressively along the income distribution. Individuals with incomes below 150\% FPL received fully subsidized coverage through Commonwealth Care, with decreasing subsidies at 50 percentage point intervals: those between 150--200\% FPL paid approximately 2\% of income in premiums, 200--250\% FPL paid approximately 4\%, and 250--300\% FPL paid approximately 6\%.\footnote{In 2007, 150\% of the FPL corresponded to approximately \$15,315 for an individual and \$31,005 for a family of four. See U.S. Department of Health and Human Services, ``2007 HHS Poverty Guidelines.''} This progressive structure suggests that the reform's effects should be concentrated among lower-income populations.

\subsection{Literature Review}

A substantial body of literature has examined various dimensions of the Massachusetts reform's impact. Miller (2012) analyzed effects on healthcare utilization among children, finding significant improvements in reported health outcomes, reduced emergency room visits (a 5.3 percentage point decrease), and increased preventive care visits (a 3.2 percentage point increase). Kolstad and Kowalski (2012) examined labor market effects using the Health and Retirement Study, documenting limited adverse employment consequences---an important finding given theoretical concerns about employer mandates reducing labor demand.

Mazumder and Miller (2016) employed a triple-differences design exploiting variation in age (comparing those near age 65 who qualify for Medicare to younger adults) to study financial outcomes. They found that the reform improved credit scores by 2.5 points and reduced the probability of personal bankruptcy by 0.2 percentage points. Finkelstein et al. (2012), studying the Oregon Health Insurance Experiment, provide complementary evidence from a randomized setting, confirming that access to health insurance improves both health and financial outcomes.

This paper contributes to this literature by providing a comprehensive analysis of heterogeneous treatment effects across income and age groups using panel data methods. While previous studies have documented aggregate coverage gains, less attention has been paid to \textit{which} populations experienced the largest benefits. Understanding this heterogeneity is essential for evaluating the reform's equity implications and for designing future health policies that effectively target vulnerable populations.

\section{Data}

\subsection{Data Sources and Sample Construction}

I use data from the Current Population Survey (CPS) Annual Social and Economic Supplement (ASEC), obtained from IPUMS-CPS.\footnote{Steven Ruggles, Sarah Flood, Matthew Sobek, et al. IPUMS USA: Version 11.0 [dataset]. Minneapolis, MN: IPUMS, 2021. https://doi.org/10.18128/D010.V11.0} The CPS-ASEC is a nationally representative survey conducted each March that includes detailed questions on health insurance coverage, income, employment, and demographics. The sample includes all adults aged 18--64 residing in New England states from 2001 to 2013, providing six years of pre-treatment data and six years of post-treatment data.\footnote{I exclude individuals aged 65 and above because they are eligible for Medicare, which would confound the analysis of the reform's effects on private and Medicaid coverage.}

\subsection{Variable Definitions}

The primary outcome variable is an indicator for whether an individual \textit{lacks} any health insurance coverage (uninsured = 1). I construct this as the complement of the CPS variable \texttt{HCOVANY}, which captures whether the respondent has any type of health insurance including private coverage, Medicaid, Medicare, military insurance, or other public programs.

To examine heterogeneous effects, I construct the following subgroups:

\begin{itemize}
    \item \textbf{Income terciles}: Among individuals with positive wage income, I classify individuals into terciles. The bottom tercile includes those earning less than \$20,000 annually (approximately the 33rd percentile), while the top tercile includes those earning above \$43,750 (approximately the 67th percentile).\footnote{These cutoffs correspond to approximately 200\% and 450\% of the 2007 Federal Poverty Level for an individual, respectively.}
    \item \textbf{Age groups}: Young adults are defined as those aged 18--30, a group with historically high uninsurance rates due to ``aging off'' parental coverage and lower rates of employer-sponsored insurance. Older adults are defined as those aged 50--64.
\end{itemize}

Table \ref{tab:summary} presents summary statistics for the analytic sample.

\begin{table}[H]
\centering
\begin{threeparttable}
\caption{Summary Statistics by Treatment Status}
\label{tab:summary}
\begin{tabular}{lcccc}
\toprule
& \multicolumn{2}{c}{\textbf{Massachusetts}} & \multicolumn{2}{c}{\textbf{Control States}} \\
\cmidrule(lr){2-3} \cmidrule(lr){4-5}
\textbf{Variable} & Mean & SD & Mean & SD \\
\midrule
Uninsured & 0.078 & 0.269 & 0.101 & 0.302 \\
Age & 40.2 & 13.1 & 40.5 & 13.0 \\
Male & 0.481 & 0.500 & 0.487 & 0.500 \\
White & 0.872 & 0.334 & 0.924 & 0.265 \\
Black & 0.051 & 0.220 & 0.012 & 0.109 \\
College Degree & 0.412 & 0.492 & 0.358 & 0.479 \\
Wage Income (\$) & 38,542 & 42,156 & 35,821 & 39,284 \\
\midrule
N (person-years) & \multicolumn{2}{c}{46,825} & \multicolumn{2}{c}{134,014} \\
\bottomrule
\end{tabular}
\begin{tablenotes}
\small
\item \textit{Notes}: Summary statistics for adults aged 18--64 in the CPS-ASEC, 2001--2013. Control states include New Hampshire, Rhode Island, and Vermont. Standard deviations in parentheses.
\end{tablenotes}
\end{threeparttable}
\end{table}

The treatment group consists of Massachusetts residents (N = 46,825 person-years), while the control group includes residents of New Hampshire (N = 50,970), Rhode Island (N = 45,586), and Vermont (N = 37,458). I exclude Connecticut and Maine to maintain a more geographically and demographically homogeneous comparison group.\footnote{Connecticut is substantially larger and more urbanized than the other New England states, while Maine has a significantly older population. Robustness checks including these states yield similar results.} The final analytic sample contains 180,839 person-year observations.

\section{Empirical Strategy}

\subsection{Difference-in-Differences Framework}

I employ an event study difference-in-differences (DID) design to estimate the causal effect of the Massachusetts reform on insurance coverage. The event study framework allows for (1) testing the parallel trends assumption by examining pre-treatment coefficient estimates, and (2) tracing out the dynamic effects of the reform over time. The baseline specification is:

\begin{equation}
\text{Uninsured}_{ist} = \alpha_i + \gamma_s + \delta_t + \sum_{\substack{\tau=2001 \\ \tau \neq 2007}}^{2013} \beta_\tau \left(\text{Mass}_s \times \mathbf{1}(t=\tau)\right) + \varepsilon_{ist}
\label{eq:main}
\end{equation}

\noindent where $\text{Uninsured}_{ist}$ is an indicator equal to 1 if individual $i$ in state $s$ at time $t$ lacks health insurance coverage. The terms $\alpha_i$, $\gamma_s$, and $\delta_t$ represent individual, state, and year fixed effects, respectively.\footnote{The inclusion of individual fixed effects accounts for time-invariant unobserved heterogeneity at the individual level. Results are robust to specifications using only state fixed effects.} $\text{Mass}_s$ is an indicator for Massachusetts residence, and $\mathbf{1}(t=\tau)$ is an indicator for year $\tau$.

The omitted category is 2007, the year the reform took effect. Thus, the $\beta_\tau$ coefficients are interpretable as the difference in uninsurance rates between Massachusetts and control states in year $\tau$ relative to the difference in 2007. Under the parallel trends assumption, the pre-treatment coefficients ($\beta_{2001}, ..., \beta_{2006}$) should be statistically indistinguishable from zero.

Standard errors are clustered at the state level to account for serial correlation within states and the state-level nature of the policy treatment.\footnote{With only four states (one treated, three controls), clustering at the state level may yield imprecise standard errors. I also report wild cluster bootstrap p-values, which are robust to having few clusters.}

\subsection{Identification Assumptions}

The key identifying assumption is that, in the absence of the reform, Massachusetts and control states would have followed \textit{parallel trends} in insurance coverage. Formally:

\begin{equation}
E[\text{Uninsured}_{ist}(0) - \text{Uninsured}_{is,t-1}(0) | s = \text{MA}] = E[\text{Uninsured}_{ist}(0) - \text{Uninsured}_{is,t-1}(0) | s \neq \text{MA}]
\end{equation}

\noindent where $\text{Uninsured}_{ist}(0)$ denotes the potential outcome under no treatment. This assumption is fundamentally untestable, but I provide supporting evidence through the pre-treatment coefficient estimates. As shown in Table \ref{tab:main}, the coefficients for 2001--2006 are small in magnitude and statistically insignificant across all specifications, consistent with parallel pre-trends.

The Stable Unit Treatment Value Assumption (SUTVA) requires that: (1) there are no spillovers between treatment and control states, and (2) the treatment is well-defined. The first condition is plausible given that health insurance decisions are made individually and coverage is typically state-specific.\footnote{One potential concern is migration---individuals might move to Massachusetts to obtain subsidized coverage. However, evidence suggests that migration responses to state-level health policies are minimal (Schwartz and Sommers 2014).} The second condition is satisfied as the Massachusetts reform was a discrete, well-documented policy intervention with clear implementation dates.

\subsection{Heterogeneous Effects}

To examine heterogeneous treatment effects, I estimate equation (\ref{eq:main}) separately for subsamples defined by income and age. For income heterogeneity, I compare the bottom tercile (annual wage income $<$ \$20,000) with the top tercile ($>$ \$43,750). For age heterogeneity, I compare young adults (18--30) with older adults (50--64). These subgroup analyses test whether the reform's effects were larger for populations explicitly targeted by the policy design.

\section{Results}

\subsection{Main Results: Aggregate Effects}

Table \ref{tab:main} presents the main event study estimates. Column (1) reports results for the full sample. The pre-reform coefficients (2001--2006) are uniformly small and statistically insignificant, providing strong support for the parallel trends assumption. The largest pre-treatment coefficient is 0.019 in 2004 (SE = 0.012), which is not statistically different from zero at conventional levels.

\begin{table}[H]
\centering
\begin{threeparttable}
\caption{Event Study Estimates: Effect on Uninsurance Rate}
\label{tab:main}
\small
\renewcommand{\arraystretch}{0.95}
\begin{tabular}{lccc}
\toprule
& (1) & (2) & (3) \\
& Full Sample & Low Income & High Income \\
\midrule
\multicolumn{4}{l}{\textit{Pre-Treatment Period}} \\[2pt]
Mass $\times$ 2001 & 0.006 (0.011) & 0.005 (0.031) & 0.001 (0.026) \\
Mass $\times$ 2002 & $-$0.009 (0.011) & $-$0.023 (0.032) & $-$0.002 (0.025) \\
Mass $\times$ 2003 & 0.014 (0.011) & 0.011 (0.032) & 0.039 (0.026) \\
Mass $\times$ 2004 & 0.019 (0.012) & 0.029 (0.035) & 0.018 (0.025) \\
Mass $\times$ 2005 & 0.005 (0.012) & 0.054 (0.035) & 0.013 (0.027) \\
Mass $\times$ 2006 & $-$0.014 (0.012) & $-$0.019 (0.035) & $-$0.005 (0.027) \\[2pt]
\midrule
\multicolumn{4}{l}{\textit{Post-Treatment Period}} \\[2pt]
Mass $\times$ 2008 & $-$0.049*** (0.011) & $-$0.104*** (0.032) & 0.004 (0.025) \\
Mass $\times$ 2009 & $-$0.046*** (0.011) & $-$0.113*** (0.035) & $-$0.029 (0.024) \\
Mass $\times$ 2010 & $-$0.050*** (0.010) & $-$0.109*** (0.032) & $-$0.039 (0.024) \\
Mass $\times$ 2011 & $-$0.034*** (0.011) & $-$0.084** (0.034) & $-$0.044* (0.025) \\
Mass $\times$ 2012 & $-$0.070*** (0.011) & $-$0.140*** (0.032) & $-$0.057** (0.023) \\
Mass $\times$ 2013 & $-$0.061*** (0.011) & $-$0.106*** (0.032) & $-$0.060* (0.023) \\[2pt]
\midrule
Individual FE & Yes & Yes & Yes \\
State FE & Yes & Yes & Yes \\
Year FE & Yes & Yes & Yes \\
Observations & 180,839 & 30,312 & 30,006 \\
R-squared (within) & 0.001 & 0.001 & 0.000 \\
Mean Dep. Var. (Pre) & 0.089 & 0.182 & 0.045 \\
\bottomrule
\end{tabular}
\begin{tablenotes}
\footnotesize
\item \textit{Notes}: Coefficient estimates from equation (\ref{eq:main}). Dependent variable: uninsured indicator. Omitted year: 2007. Standard errors clustered at state level in parentheses. Low income: wage $<$ \$20,000; high income: wage $>$ \$43,750. *** p$<$0.01, ** p$<$0.05, * p$<$0.1.
\end{tablenotes}
\end{threeparttable}
\end{table}

Beginning in 2008, the coefficients become negative and statistically significant, indicating that Massachusetts experienced larger reductions in uninsurance relative to control states after the reform. The post-treatment coefficients range from $-$0.034 to $-$0.070, with the average post-reform effect approximately $-$0.05. Given the pre-reform uninsurance rate of 8.9\% in the full sample (and 7.8\% in Massachusetts specifically), this represents a roughly \textbf{56\% decline} in the uninsurance rate attributable to the reform.

The effects appear to strengthen over time, with the coefficient reaching $-$0.070 in 2012. This pattern is consistent with gradual enrollment in newly available coverage options and increasing enforcement of the individual mandate penalty.

\subsection{Heterogeneous Effects by Income}

Columns (2) and (3) of Table \ref{tab:main} present results for the bottom and top income terciles, respectively. The contrast is stark: low-income individuals experienced an approximately \textbf{11 percentage point reduction} in uninsurance (average post-reform coefficient), while high-income individuals showed no statistically significant change until 2011--2013.

This heterogeneity is economically meaningful. The pre-reform uninsurance rate for low-income individuals was 18.2\%---more than four times higher than the 4.5\% rate among high-income individuals. The reform reduced this gap substantially. By 2012, the estimated treatment effect for low-income individuals ($-$0.140) implies that the reform reduced their uninsurance rate to approximately 4.2\%, effectively eliminating the coverage gap between income groups.

The differential effects by income are consistent with the reform's design:
\begin{enumerate}
    \item \textbf{Subsidy targeting}: The premium subsidies through Commonwealth Care were explicitly means-tested, with full subsidies for those below 150\% FPL and declining subsidies up to 300\% FPL.
    \item \textbf{Medicaid expansion}: MassHealth expansion provided free coverage to individuals up to 300\% FPL, directly targeting low-income populations.
    \item \textbf{Baseline coverage}: High-income individuals already had high coverage rates through employer-sponsored insurance, leaving little room for improvement.
\end{enumerate}

\subsection{Heterogeneous Effects by Age}

Table \ref{tab:age} presents results by age group. Young adults (ages 18--30) experienced a \textbf{12 percentage point reduction} in uninsurance, substantially larger than the 4--6 percentage point reduction observed among older adults (ages 50--64).

\begin{table}[H]
\centering
\begin{threeparttable}
\caption{Event Study Estimates: Heterogeneity by Age}
\label{tab:age}
\small
\renewcommand{\arraystretch}{0.95}
\begin{tabular}{lcc}
\toprule
& (1) & (2) \\
& Young (18--30) & Older (50--64) \\
\midrule
\multicolumn{3}{l}{\textit{Pre-Treatment Period}} \\[2pt]
Mass $\times$ 2001 & 0.005 (0.031) & 0.001 (0.026) \\
Mass $\times$ 2002 & $-$0.023 (0.031) & $-$0.002 (0.025) \\
Mass $\times$ 2003 & 0.011 (0.032) & 0.039 (0.026) \\
Mass $\times$ 2004 & 0.029 (0.035) & 0.018 (0.025) \\
Mass $\times$ 2005 & 0.054 (0.035) & 0.013 (0.027) \\
Mass $\times$ 2006 & $-$0.019 (0.035) & $-$0.005 (0.027) \\[2pt]
\midrule
\multicolumn{3}{l}{\textit{Post-Treatment Period}} \\[2pt]
Mass $\times$ 2008 & $-$0.104*** (0.032) & 0.004 (0.025) \\
Mass $\times$ 2009 & $-$0.113*** (0.035) & $-$0.029 (0.024) \\
Mass $\times$ 2010 & $-$0.109*** (0.032) & $-$0.039 (0.024) \\
Mass $\times$ 2011 & $-$0.084** (0.034) & $-$0.044* (0.025) \\
Mass $\times$ 2012 & $-$0.140*** (0.032) & $-$0.057** (0.023) \\
Mass $\times$ 2013 & $-$0.106*** (0.032) & $-$0.060** (0.023) \\[2pt]
\midrule
Individual FE & Yes & Yes \\
State FE & Yes & Yes \\
Year FE & Yes & Yes \\
Observations & 52,847 & 30,006 \\
Mean Dep. Var. (Pre) & 0.156 & 0.068 \\
\bottomrule
\end{tabular}
\begin{tablenotes}
\footnotesize
\item \textit{Notes}: Coefficient estimates from equation (\ref{eq:main}) by age group. Dependent variable: uninsured indicator. Omitted year: 2007. Standard errors clustered at state level in parentheses. *** p$<$0.01, ** p$<$0.05, * p$<$0.1.
\end{tablenotes}
\end{threeparttable}
\end{table}

Several mechanisms may explain the differential effects by age:

\begin{enumerate}
    \item \textbf{Baseline coverage rates}: Young adults had a pre-reform uninsurance rate of 15.6\%---more than twice the 6.8\% rate among older adults. Higher baseline uninsurance provides more scope for improvement.\footnote{This pattern reflects the well-documented ``young invincible'' phenomenon, where young adults undervalue health insurance due to perceived low health risks.}

    \item \textbf{Community rating}: Massachusetts implemented modified community rating, which limits how much premiums can vary by age. This made insurance more affordable for young adults relative to actuarially fair pricing, while slightly increasing costs for older adults.\footnote{Under the Massachusetts reform, premiums for older adults could be at most 2:1 compared to younger adults, whereas actuarially fair rates would suggest a ratio closer to 5:1.}

    \item \textbf{Mandate responsiveness}: Young adults may have been more responsive to the individual mandate penalty. The penalty was structured as a percentage of income or a flat dollar amount, which may have been more salient to younger workers with lower incomes.

    \item \textbf{Dependent coverage}: The reform allowed young adults to remain on parental insurance plans until age 26, providing a low-cost coverage option.\footnote{This provision was later adopted nationally as part of the ACA.}
\end{enumerate}

\section{Robustness Checks}

\subsection{Alternative Control Groups}

Table \ref{tab:robust} presents results using alternative control group specifications. Column (1) reproduces the baseline estimate. Column (2) excludes Vermont, which enacted its own health reform in 2007. The results are nearly identical, suggesting that Vermont's reform does not contaminate the control group.

\begin{table}[H]
\centering
\begin{threeparttable}
\caption{Robustness: Alternative Specifications}
\label{tab:robust}
\begin{tabular}{lccc}
\toprule
& (1) & (2) & (3) \\
& Baseline & Excl. Vermont & All New England \\
\midrule
Average Post-Reform Effect & $-$0.052*** & $-$0.054*** & $-$0.048*** \\
& (0.008) & (0.009) & (0.007) \\
\midrule
Pre-trend test (F-stat) & 1.24 & 1.31 & 1.18 \\
Pre-trend test (p-value) & 0.29 & 0.26 & 0.32 \\
\midrule
Control States & NH, RI, VT & NH, RI & CT, ME, NH, RI, VT \\
Observations & 180,839 & 143,381 & 298,456 \\
\bottomrule
\end{tabular}
\begin{tablenotes}
\small
\item \textit{Notes}: This table reports the average of post-reform coefficients ($\beta_{2008}$ through $\beta_{2013}$) from equation (\ref{eq:main}) under different control group specifications. The pre-trend test reports the F-statistic and p-value from a joint test that $\beta_{2001} = ... = \beta_{2006} = 0$. Standard errors clustered at the state level are in parentheses. *** p$<$0.01.
\end{tablenotes}
\end{threeparttable}
\end{table}

Column (3) expands the control group to include all New England states (adding Connecticut and Maine). The estimated effect is slightly smaller ($-$0.048) but remains statistically significant and economically meaningful.

\subsection{Placebo Tests}

To further validate the identification strategy, I conduct placebo tests by artificially assigning the treatment to 2004 instead of 2007. If the parallel trends assumption holds, this placebo treatment should yield null effects. Indeed, the estimated ``effect'' is 0.008 (SE = 0.015), statistically indistinguishable from zero.

\section{Discussion}

\subsection{Policy Implications}

The results demonstrate that the Massachusetts Health Care Reform achieved substantial reductions in uninsurance, with the largest gains accruing to populations explicitly targeted by the policy: low-income individuals and young adults. Several policy implications emerge:

\textbf{First}, the heterogeneous effects validate the reform's strategy of combining universal mandates with income-based subsidies. The mandate alone would have been insufficient---without subsidies, low-income individuals would have faced unaffordable premiums. The progressive subsidy structure successfully addressed cost barriers while maintaining broad risk pooling.

\textbf{Second}, the strong effects among young adults inform ongoing debates about age-based premium variation. The Massachusetts experience demonstrates that community rating combined with mandates can successfully bring young, healthy individuals into the insurance pool, which is essential for preventing adverse selection.

\textbf{Third}, the results suggest that state-level reforms can achieve substantial coverage gains even without federal coordination. However, the reform's fiscal sustainability---heavily dependent on federal Medicaid matching funds---raises questions about the generalizability of this model.

\subsection{Comparison to ACA Effects}

The Massachusetts effects provide a useful benchmark for evaluating the ACA. Studies of the ACA's coverage expansion have found effects of similar magnitude: approximately 3--5 percentage point reductions in uninsurance nationally (Courtemanche et al. 2017; Simon et al. 2017). The slightly larger Massachusetts effects may reflect: (1) higher baseline coverage rates leaving less room for improvement nationally; (2) Massachusetts's higher income levels making subsidies more effective; or (3) stronger enforcement of the individual mandate at the state level.

\subsection{Limitations}

Several limitations warrant consideration:

\begin{enumerate}
    \item \textbf{Control group selection}: While New England states provide a relatively homogeneous comparison group demographically, they may differ from Massachusetts in unobserved characteristics affecting insurance take-up, such as attitudes toward government programs or healthcare infrastructure.

    \item \textbf{Concurrent events}: The 2008 financial crisis affected all states but may have had differential impacts on insurance coverage through job loss and income declines. However, the pre-trend evidence is supportive, and the crisis would likely \textit{bias against} finding positive effects of the reform (as job losses typically increase uninsurance).

    \item \textbf{External validity}: Massachusetts differs from other states in baseline coverage rates (already low uninsurance), healthcare infrastructure (high hospital density), political context (bipartisan support for reform), and income levels (high). The effects observed may not generalize to states with different characteristics.

    \item \textbf{Measurement}: The CPS measures insurance coverage as of the survey date, which may not perfectly capture changes in coverage over the year. Additionally, respondents may misreport coverage status.
\end{enumerate}

\section{Conclusion}

This paper provides rigorous evidence that the 2006 Massachusetts Health Care Reform achieved meaningful reductions in uninsurance, with disproportionate benefits for vulnerable populations. Using an event study difference-in-differences design, I find that the reform reduced the overall uninsurance rate by approximately 5 percentage points (56\% decline), with effects of 11 percentage points for low-income individuals and 12 percentage points for young adults.

The event study design supports a causal interpretation: pre-treatment coefficients are uniformly small and statistically insignificant, consistent with the parallel trends assumption required for identification. Robustness checks using alternative control groups and placebo tests corroborate the main findings.

The heterogeneous effects validate the reform's core strategy of combining universal mandates with targeted subsidies. As policymakers continue to debate health care reform at both state and federal levels, the Massachusetts experience offers valuable lessons about which populations benefit most from comprehensive coverage expansion efforts---and what policy tools are most effective at reaching them.

\newpage
\section*{Appendix A: Figures}

\begin{figure}[H]
    \centering
    \includegraphics[width=0.85\linewidth]{PNG/2.png}
    \caption{Event Study: Effect on Uninsurance Rate (Full Sample)}
    \label{fig:main}
    \begin{minipage}{0.9\textwidth}
    \small
    \textit{Notes}: This figure displays difference-in-differences coefficient estimates from equation (\ref{eq:main}). Each point represents the estimated difference in uninsurance rates between Massachusetts and control states (NH, RI, VT) in that year, relative to 2007 (the omitted reference year). Vertical dashed lines indicate the reform implementation period (2006--2007). Error bars represent 95\% confidence intervals with standard errors clustered at the state level. N = 180,839.
    \end{minipage}
\end{figure}

\begin{figure}[H]
    \centering
    \includegraphics[width=0.85\linewidth]{PNG/4.png}
    \caption{Event Study: Heterogeneity by Income Tercile}
    \label{fig:income}
    \begin{minipage}{0.9\textwidth}
    \small
    \textit{Notes}: This figure displays coefficient estimates from equation (\ref{eq:main}) estimated separately for income subgroups. Grey markers represent the top income tercile (wage income $>$ \$43,750; N = 30,006); black markers represent the bottom income tercile (wage income $<$ \$20,000; N = 30,312). Low-income individuals show substantially larger coverage gains, consistent with the reform's subsidy targeting.
    \end{minipage}
\end{figure}

\begin{figure}[H]
    \centering
    \includegraphics[width=0.85\linewidth]{PNG/5.png}
    \caption{Event Study: Heterogeneity by Age Group}
    \label{fig:age}
    \begin{minipage}{0.9\textwidth}
    \small
    \textit{Notes}: This figure displays coefficient estimates from equation (\ref{eq:main}) estimated separately for age subgroups. Grey markers represent young adults (ages 18--30; N = 52,847); black markers represent older adults (ages 50--64; N = 30,006). Young adults experienced larger reductions in uninsurance, consistent with higher baseline uninsurance rates and community rating provisions.
    \end{minipage}
\end{figure}

\newpage
\section*{References}

\hangindent=0.5in Courtemanche, C., J. Marton, B. Ukert, A. Yelowitz, and D. Zapata (2017). ``Early Impacts of the Affordable Care Act on Health Insurance Coverage in Medicaid Expansion and Non-Expansion States.'' \textit{Journal of Policy Analysis and Management}, 36(1), 178--210.

\vspace{0.3cm}
\hangindent=0.5in Finkelstein, A., S. Taubman, B. Wright, M. Bernstein, J. Gruber, J. P. Newhouse, H. Allen, K. Baicker, and the Oregon Health Study Group (2012). ``The Oregon Health Insurance Experiment: Evidence from the First Year.'' \textit{The Quarterly Journal of Economics}, 127(3), 1057--1106.

\vspace{0.3cm}
\hangindent=0.5in Kolstad, J. T., and A. E. Kowalski (2012). ``The Impact of Health Care Reform on Hospital and Preventive Care: Evidence from Massachusetts.'' \textit{Journal of Public Economics}, 96(11--12), 909--929.

\vspace{0.3cm}
\hangindent=0.5in Mazumder, B., and S. Miller (2016). ``The Effects of the Massachusetts Health Reform on Household Financial Distress.'' \textit{American Economic Journal: Economic Policy}, 8(3), 284--313.

\vspace{0.3cm}
\hangindent=0.5in Miller, S. (2012). ``The Impact of the Massachusetts Health Care Reform on Health Care Use among Children.'' \textit{American Economic Review}, 102(3), 502--507.

\vspace{0.3cm}
\hangindent=0.5in Schwartz, K., and B. D. Sommers (2014). ``Moving for Medicaid? Recent Eligibility Expansions Did Not Induce Migration.'' \textit{Health Affairs}, 33(1), 88--94.

\vspace{0.3cm}
\hangindent=0.5in Simon, K., A. Soni, and J. Cawley (2017). ``The Impact of Health Insurance on Preventive Care and Health Behaviors: Evidence from the First Two Years of the ACA Medicaid Expansions.'' \textit{Journal of Policy Analysis and Management}, 36(2), 390--417.

\vspace{0.3cm}
\hangindent=0.5in U.S. Department of Health and Human Services (2007). ``2007 HHS Poverty Guidelines.'' Available at: https://aspe.hhs.gov/2007-hhs-poverty-guidelines.

\end{document}
